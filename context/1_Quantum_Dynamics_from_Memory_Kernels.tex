\documentclass[preprint]{revtex4-2}

\usepackage{mathtools, amssymb, fixdif, derivative, physics2, bm, bbm}
\usephysicsmodule{ab, ab.braket, diagmat, xmat}
\usepackage{lmodern}
\usepackage[mono = false]{libertine}
\usepackage[hidelinks, colorlinks]{hyperref}
\makeatletter
\let\oldHyPsd@CatcodeWarning\HyPsd@CatcodeWarning
\renewcommand \HyPsd@CatcodeWarning [1]%
  {\ifnum\pdfstrcmp{#1}{math shift}=0 \else \oldHyPsd@CatcodeWarning{#1} \fi}
\pdfstringdefDisableCommands{\let\HyPsd@CatcodeWarning\@gobble}
\makeatother
\newtheorem{example}{Example}
\DeclareRobustCommand \iu {\mathrm i}
\DeclareRobustCommand \upe {\mathrm e}
\usepackage{graphicx}

\begin{document}

\title{MKCT --- Quantum Dynamics from Memory Kernels}
\author{Mingyu Xia}
\affiliation{Laboratory of Theoretical and Computational Chemistry,
  Westlake University}
\email{xiamingyu@westlake.edu.cn}
\date{Released 2025-12-01}

\maketitle
\tableofcontents

\section{Quantum dynamics, correlation functions and memory kernels}

\clearpage

\section{Approximation of $\mathsf{K_{n+1}(t)}$ and the truncation
  of MKCT}

\subsection{Pad\'e approximations}

A function can be expanded into a series
\begin{equation}
  y(x) \approx \sum_n^N \frac{y^{(n)}}{n!} x^n = \sum_n^N c_n x^n
\end{equation}
For the Taylor expansion, it can converge slowly.
Pad\'e constructs a rational function $R(x)$
\begin{equation}
  R(x) = \frac{a_0 + a_1x + \cdots + a_m x^m}{1 + b_1x + \cdots + b_nx^n},
\end{equation}
such that the derivatives $R^{(n)} = f^{(n)}$, and Pad\'e approximation usually
converge faster than that of Taylor.
\begin{example}
  Taylor and Pad\'e approximation for exponent.
  \[
    y(x) = \exp[-(2 + 2\iu)x] = \upe^{-\alpha x}
  \]
  The top 3, 5, 7, 8, 9, 11, 13 terms approximation of Taylor expansion and
  Pad\'e are plotted as (\textsf{Python} source is avaliable at
  \href{https://github.com/myhsia/MKCT/blob/main/source/Taylor_vs_Pade_Exponent.py}
  {GitHub})
  \begin{figure}[htbp]
    \begin{minipage}{.48\linewidth}
      \centering
      \includegraphics[width = \linewidth, page = 1]
        {../source/Taylor_vs_Pade_Exponent.pdf}
    \end{minipage}
    \hfill
    \begin{minipage}{.48\linewidth}
      \centering
      \includegraphics[width = \linewidth, page = 2]
        {../source/Taylor_vs_Pade_Exponent.pdf}
    \end{minipage}
    \caption{Converge trends between Taylor and Pad\'e approximations}
  \end{figure}
\end{example}

\end{document}